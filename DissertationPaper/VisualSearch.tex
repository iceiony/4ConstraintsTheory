\documentclass{article}
\usepackage{graphicx}
\usepackage{hyperref}
\hypersetup{colorlinks,urlcolor=blue}
\let \shorttitle \textbf
\begin{document}


\section{Visual Search}

\subsection{Motivation}
\shorttitle{Visual Search}

In optimising our tool positional search, we must determine the regions of contact that are likely to be good locations for the objects interaction.
The proposal follows the idea mentioned by \cite{battaglia2013}, in that humans build mental models of objects to allow inference over the physical world.
Human subjects would have an understanding over the geometric constraints of the physical world and of the forces generated by their actions.

Tool and object interaction baring high likelihood of success, would have to satisfy two criteria: 
\begin{enumerate}
\item \textbf{Geometric constraints} of the two objects must be met ( i.e. the shapes of the two objects should adequately fit )
\item \textbf{Contact forces} generated must correspond to the intention of the actions executed 
  (i.e. for a lifting action, the human subject would focus on points of interaction that would permit vertical forces to be applied)
\end{enumerate}

The total space of possible solutions can be reduced by these criteria without a physics simulation (akin mental simulations \cite{osiurak2014}).
The potential of likely solutions can be later verified using a physics engine. 

\subsection{Geometric Constraints}
The geometries of the tool and object must be compatible. In other words, either the active or passive object's geometry must fit within the gaps and edges of its counterpart.
Tighter fits, offer better transfer of energy and control over the movement of the passive object. A good fit would therefore require less manipulation effort.
In the paradigm of 4CT \cite{osiurak2014}, the effort constraint may explain user's preference for one geometric solution over another.

It is not necessary for the whole geometry of objects to fit. 
It is sufficient that the tool have the necessary parts to act as affordance and functional basis (end effector\cite{zhu2015}).
If the functional part matches the gaps of the passive object, then the configuration is likely to achieve the desired effect. 

\subsubsection{Shape Description and Matching}
We consider techniques of shape similarity in matching object parts.
Shape analysis techniques have traditionally been engaged in image processing and robot vision for detecting and tracking objects of similar features.
The last two decades have provided many approaches for this type of problem \cite{loncaric1998,zhang2004,veltkamp2001,robert2012}.
Fundamentally, the challenge lies in detecting similarity even as objects undergo geometric transformations (i.e. rotation,translation,scaling and shearing).     

\cite{zhang2004} classify shape similarity into contour-based and region-based techniques. These are further divided into structural and global approaches.
These are further divided into structural and global approaches.
A comprehensive list and relations can be found in fig. \ref{fig:shape_similarity}. 

\begin{figure}[b]
  \centering
  \includegraphics[width=1\textwidth]{./figures/similarity_techniques.png}
  \caption{Classification of shape similarity techniques (reprinted from \cite{zhang2004})}
  \label{fig:shape_similarity}
\end{figure}  

Contour techniques asses shape similarity by extracting features from the edge of detected objects.
In comparison, region techniques work by assessing surface level information such as: colour features, gradient changes and surface medial.
Techniques from both approaches have justification in human perception \cite{chatbri2016}.
Nonetheless, our use case excludes most region-based approaches as tool parts must correspond to shape gaps.
It is hard to consider gaps as having the surface information needed for region-based matching.

In structural approaches, shapes are considered as composed out of primitives. In the case of contour techniques, primitives are segments on the boundary of an object.
The organisation of primitives can be linear (feature vector\cite{zhang2004}) or hierarchical (tree like structures\cite{zhu2015}).
Two objects are considered similar when they have the same primitive structures (or features).
In comparison, global approaches make use of shapes as a whole when assessing similarity. 

Both structural and global approaches have justification in human perception \cite{zhang2004}.
Human subjects show a preference for features even when other shape descriptors are available \cite{chatbri2016}.
At the same time, global shape perception seems to preceed local feature detection\cite{navon1977}. 

Matching human behaviour requires more insight into human visual perception.
Loncaric\cite{loncaric1998} describes some theories of human visual perception with interest in image processing.
In a tool use scenario, emphasis should be given to theories describing perception as volumetric, such as through the use of generalised cylinders (geons\cite{dickinson2014}).
Such insight may better explain human tool performance, but will however remain for future work.    

\subsubsection{Limitations}
As previously remarked, contour based techniques contain promising approaches for solving tool-object matching problems.
It is important to consider some of the limitations of such techniques, especially with regards to human ability. 

Global contour matching techniques are sensitive to occlusion (i.e. part of the object is hidden from the view point by some obstacle).
In such cases, structural approaches are better suited to identify shapes from visible parts.
Human perception is regarded as able to recognise objects even from sparse information or occluded perspective \cite{loncaric1998}.
In solving tool use scenarios, it is important to consider a subject's visual perspective ( point of view ).
The geometry of objects may not be fully visible. 

Most shape similarity techniques are tailored to analysing image based information (i.e. 2D projections of reality).
Human perception however also makes use of depth related knowledge.
As tool use encompasses real world geometric constraints, any contour based approach would have to be adaptable to 3D information ( e.g. point clouds ).
To simplify the problem of visual analysis, surface point coordinates are extracted directly from the physics simulator. 

We next consider a novel and simple approach for matching point surfaces. It avoids most of the problems enumerated above and scales well to multi-dimensional data.  

\end{document}
