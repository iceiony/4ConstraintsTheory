\documentclass[
    a4paper,
    man,
    donotrepeattitle,
    floatsintext,
    british
]{apa6}

\usepackage{standalone}
\usepackage{newclude}

\usepackage[british]{babel}
\usepackage[utf8]{inputenc}
\usepackage{epstopdf}
\usepackage{csquotes}
\usepackage[hidelinks]{hyperref}
\usepackage[
    style=apa,
    backend=biber,
    sortcites=true,
    sorting=nyt,
%    isbn=false,
%    url=false,
%    doi=false,
%    eprint=false,
%    hyperref=false,
%    backref=false,
%    firstinits=false,
]{biblatex}

\DeclareLanguageMapping{british}{british-apa}

% maps apacite commands to biblatex commands
\let \citeNP \cite
\let \citeA \textcite
\let \cite \parencite

\bibliography{references}

\usepackage{amsmath}
\usepackage{graphicx}
\usepackage{hyperref}
\usepackage{subcaption}
\hypersetup{colorlinks,urlcolor=blue}

\usepackage{listings}
\usepackage{xcolor}
\definecolor{darkgray}{rgb}{.35,.25,.35}
\definecolor{lightgray}{rgb}{.6,.6,.6}

\lstdefinelanguage{bash}{
  basicstyle=\scriptsize\ttfamily\color{white},
  backgroundcolor=\color{lightgray},
  commentstyle = \color{darkgray},
  keywordstyle = \color{white},
  rulecolor    = \color{black},
  stringstyle  = \color{white}
  commentstyle=\color{lightgray}\ttfamily
}

\lstset{
   language=bash,
   extendedchars=true,
   basicstyle=\scriptsize\ttfamily,
   showstringspaces=false,
   showspaces=false,
   tabsize=2,
   showtabs=false,
   captionpos=b,
}

\usepackage{lipsum} 

\title{SPATIAL REASONING IN HUMAN TOOL USE}
\shorttitle{}
\author{
  \small{by}\\
  Adrian Ionita (ID: 1057404)
}
\affiliation{Supervisor: Dietmar Heinke\\
\vfill
\includegraphics[width=40mm]{figures/bham_logo.png}
\vfill
Submitted in partial fulfilment\\
of the requirements for the degree of\\ 
Master of Science
\vfill
School of Psychology\\
University of Birmingham\\
Birmingham, UK\\
September 2016
}
\abstract{
Tool use is a fundamental human ability that distinguishes man from other species.
This project investigates tool use reasoning in healthy individuals through the use of a computational model.
The focus is on solving geometric constraints in a novel tool use case.
Two approaches are considered for a computational model.
An exhaustive search, that finds configurations of tool-object interactions through searching of all possible alternatives. 
A heuristic search model, that reduces search space using visual queues of object and tool parts.
The heuristic model employs a novel surface similarity technique in assessing object parts which have complementary geometric features. 
Both exhaustive and heuristic models have the potential for simulating human behaviour, although further development is required.
}

\begin{document}
\pagenumbering{gobble}
\thispagestyle{otherpage}
\maketitle

\section{\normalfont Acknowledgements}
I would like to thank Dietmar Heinke, project supervisor, for his guidance and making tool-use a very interesting topic to work on. 
Thanks also go to Fran\c{c}ois Osiurak, the author of the four constraints theory, for his availability to clarifying concepts from his work. 
Last but not least, I would thank Rhiannon Davies for being the first subjected to proof-reading my work and doing a heck of a job at it.
\clearpage

\pagenumbering{arabic}

\tableofcontents
\clearpage
\listoffigures
\clearpage

\include*{ExhaustiveSearch}
\include*{HeuristicSearch}
\section{Conclusion}
\shorttitle{Conclusion}

This project has investigated two different approaches for a computational model of human tool-use reasoning.
A novel tool-use experiment was used to assess model performance.
As the experiment requires solving geometric constraints, the model's theoretical basis is best captured by the technical reasoning process defined in 4CT. 

Physics engines are a potential approach for finding fitting solution, by searching through all possible tool-object configurations.  
Although exhaustively searching return valid results, the model proves computationally unfeasible and unrealistic of human ability. 
The model requires long execution times and would find solutions that are not physically possible to reach. 

Heuristic approaches were considered for a second model based on visual cues of object and tool parts.  
The underlying hypothesis was that parts which fit would share similar geometries. 
A novel surface similarity technique was presented to assess object parts from 3D data points.
Due to time constraints, the approach was not fully implemented into a model demonstrating fitting solutions.
Only the visual search step of the process was implemented, but requires further verification.

Both the exhaustive and heuristic search models can be improved to provide better results.
The exhaustive search's simulation time can be reduced by running code on multiple machines and aggregating results.
The heuristic model can be improved through better analysis of objects geometries, but also by considering forces generated by user actions (i.e. lifting).
Nevertheless,even with the above improvements, models results require formal comparison to observed human behaviour performing a similar tool-use experiment.  

\clearpage

\include*{Appendix}
\shorttitle{}
\printbibliography
\end{document}

