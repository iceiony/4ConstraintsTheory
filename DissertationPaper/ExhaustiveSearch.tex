\documentclass[11]{article}

\usepackage{graphicx}
\usepackage{hyperref}
\usepackage{subcaption}
\usepackage{amsmath}
\usepackage[
    backend=bibtex,
%    isbn=false,
%    url=false,
%    doi=false,
%    eprint=false,
%    hyperref=false,
%    backref=false,
%    firstinits=false,
]{biblatex}
\bibliography{references}

\hypersetup{colorlinks,urlcolor=blue}
\let \shorttitle \textbf
\begin{document}

\section{Introduction}
Tool use is one of the human abilities that uniquely distinguishes man from other species. 
We refer to tools as hand held devices used in making changes to the surrounding environment. 
Cases of tool use have been reported in other species, but never to the extent engaged by humans \cite{boysen1999,harrington2009,lefebvre2004}. 
As a defining attribute of human cognition, we wish to lay the foundations for a computational model of tool use reasoning.

Our model stems from the architectural framework defined by the four constraints theory (4CT) \cite{osiurak2014a}.
In the following we describe 4CT with its implications in forming a computational model.  

\subsection{The four constraints theory}
4CT defines tool use behaviour of healthy individuals based on the empirical investigation of aprxia.
Apraxia is a neurological disorder impairing a person's ability to plan and execute sequences of movements.
The term covers a multitude of symptoms and levels of severity (e.g. dyspraxia, ideomotor aparaxia, apraxia of speech).
The underlying cause however is physical damage to the left hemisphere of the brain\cite{osiurak2013}.

\begin{figure}[h]
  \centering
  \includegraphics[width=.9\textwidth]{./figures/4CTArchitecture.png}
  \caption{4CT architecture reprinted from \cite{osiurak2014a}}
  \label{fig:4CTArchitecture}
\end{figure}      

Unlike competing theories, 4CT distinguishes between a conceptual and a production system when describing tool use(fig. \ref{fig:4CTArchitecture}).
Main focus is given to the conceptual system which encapsulates cognitive reasoning.
Further, the production system dictates movement by transforming a person's intent into motor control. 
By this distinction, 4CT's conceptual system is a suitable candidate for a model concentrated on reasoning.

%-----------
% More paragraphs needed to explain why 4CT is desired over other theories
%-----------

The theory characterises tool use situations as problem solving tasks requiring strong cognitive abilities. 
The four constraints of \emph{mechanics, space, time,} and \emph{effort} are the dimensions within which problems are defined.
Aiding problem solving are four cognitive processes: \emph{technical reasoning, semantic reasoning, working memory,} and \emph{simulation based decision making} (fig. \ref{fig:4CTArchitecture}).  
Even a simple scenario like slicing bread can be seen as reasoning about the knife's sharpness,length,serration and the movements necessary to manifest cutting effects. 

Problem solving becomes more apparent in the absence of familiar tools, when subjects are required to repurpose tools or fashion new ones.
A distinction is made between novel tool use and familiar tool use. 
Semantic reasoning refers to situations when subjects are accustomed with a tool's common purpose. 
For example a knife is used for cutting bread. 
The association is made on prior experience without considering the knife's physical properties.  
In novel cases however, technical reasoning is the inference of how the tool's properties can solve problems.
In the absence of a bread knife a saw would be a better cutting device than a spreading knife. 
Technical reasoning is a more general process for solving problems, but is more cognitively involved.  

The effort constraint and simulation based decision making refer to tool use energy cost. 
A simple rule of survival is that actions should require less energy than the rewards gained \cite{proffitt2006}.
Perceived effort would explain user's preference for one tool use method over another. 
Even when multiple tools are available, differences in their physical properties can lead to differences in effort which dictate choice. 

Working memory is required in the context where multiple steps must be fulfilled to achieve a goal. 
It describes a subject's ability to split activities into sub-goals and hold them in memory. 
Complex tasks such as fixing a radio would involve multiple steps and tool selection. 
However, such complex objectives are beyond the scope of this project.

A computational model should initially solve simple tool-use problems before complex ones.
Our focus is therefore on single step tool and object interaction involving generic situations.
These situations would be best solved through technical reasoning for its general problem solving abilities. 
However, beyond simple attributes such as object length,sharpness,abrasiveness, 4CT is not able to explain human decision-making.
Crucial factors of forces and geometric constraints are simply abstracted as mechanical knowledge. 
A computational model would therefore have to elaborate these items before fitting into the larger 4CT framework. 
The topic of this paper is solving geometric constraints through spatial reasoning in order to achieve simple tool-object interaction.  

\section{Experimental Setup}

Dietmar Heinke and Fran\c{c}ois Osiurak,the author of 4CT, have devised an experimental setup that tests human ability to use novel objects. 
The experiment is intended for use in both a computational model and to investigate human reasoning on tool and object geometries. 
Human trials were run in parallel to this project but do not constitute part of it. 
Nevertheless, this paper will refer to observed human behaviour even though data has not yet been published. 

\begin{figure}[!h]
  \centering
  \begin{subfigure}{0.49\textwidth}
    \includegraphics[width=1\linewidth]{./figures/obj51.png}
    \caption{Passive object}
    \label{fig:obj51}
  \end{subfigure}
  \begin{subfigure}{0.49\textwidth}
    \includegraphics[width=1\linewidth]{./figures/obj52.png}
    \caption{Active object}
    \label{fig:obj52}
  \end{subfigure}
  \caption{Pair of novel tool and object models}
\end{figure}

A human subject is presented two novel objects composed out of LEGO blocks.
One object is passive and may not be directly touched (fig. \ref{fig:obj51}).
The second object is active and must be used to lift the passive object to a new location (fig. \ref{fig:obj52}).
As a tool use task, the active and passive objects are analogous to tool and target object.

Due to the unusual geometries, subjects must reason about the correct spatial fitting of tool and object.    
Formally solving object-tool fitting requires considerations of force closure.
However, a shortcut to computing forces and physical interaction is to consider the use of a physics engine. 
To some extent, it is believed that the human brain computes approximate Newtonian laws \cite{battaglia2013}. 
Similarly, a physics engines does not compute exact friction, closure or gravity laws, but provides realistic alternative results. 
A tool use model can employ physics engines for one of multiple reasons:
\begin{enumerate}
      \item the engine can serve as source of inference over physical laws   
      \item it can verify solutions of tool-object fitting through simulation
      \item it is analogous to mental simulation (4CT)
\end{enumerate}

\section{Physics Engine Review}
For the model we consider physics engines written in C++ or available to Matlab. 
These engines should preferably be free or have reduced licensing fees. 
Unfortunately, the majority of engines dedicated to scientific simulation enquire large fees (e.g. Vortex , MuJoCo).
Most literature reviews focus on commercial or open source software dedicated to game development \ref{boeing2007,roennau2013,hummel2012}. 
In an academic context, physics engines have traditionally been used in robotic simulations. 


\printbibliography
\end{document}
